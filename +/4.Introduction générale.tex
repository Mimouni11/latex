





\chapter*{Introduction générale  \markboth{Introduction générale} {}}
%\markboth {General Introduction}{}}
%\addcontentsline{toc}{chapter}{General Introduction}

Dans le contexte actuel de la gestion d'entreprise axée sur l'efficacité opérationnelle et la rentabilité, l'optimisation des ressources et des actifs est devenue un impératif stratégique pour les organisations modernes. Parmi ces actifs essentiels, les parcs d'entreprise, qu'ils comprennent des véhicules, des flottes de transport ou d'autres équipements, représentent des investissements substantiels nécessitant une gestion rigoureuse et proactive.\\

Face à cette exigence croissante, notre projet s'attache à répondre aux besoins de gestion des parcs d'entreprise à travers le développement d'une application mobile novatrice et fonctionnelle, intégrant les principes de la GMAO (Gestion de Maintenance Assistée par Ordinateur). Cette solution numérique vise à rationaliser les processus de gestion en offrant aux parties prenantes une plateforme intégrée pour surveiller, entretenir et optimiser les parcs d'entreprise de manière efficace et transparente.\\

Dans le cadre de ce rapport, nous présentons une analyse approfondie du développement de cette application, mettant en lumière nos objectifs stratégiques, les fonctionnalités clés de l'application, ainsi que les défis rencontrés et les stratégies d'atténuation mises en œuvre. Nous explorerons également les implications potentielles de cette solution pour les entreprises utilisatrices, notamment en termes d'amélioration de l'efficacité opérationnelle, de réduction des coûts et d'optimisation des performances.







